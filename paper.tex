\documentclass{vldb}
\usepackage{graphicx}
\usepackage{balance}

\newcommand{\tbr}{\textbf{[TO BE REVISED]}}
\newcommand{\tbd}{\textbf{[TO BE DONE]}}

\begin{document}

% title
\title{A Survey on Ordered Binary Decision Diagrams}

% author(s)
\numberofauthors{1}

\author{
\alignauthor
	Daniel Kocher\\
    \affaddr{University of Salzburg}\\
    \email{Daniel.Kocher@stud.sbg.ac.at}
}

\maketitle

\begin{abstract}
This paper serves as a survey on \textit{ordered binary Decision diagrams}
(\textit{OBDDs}). It provides an overview of the representation of OBDDs, the
operations which can be applied efficiently, some aspects when implementing them
as well as their limitations and how to probably overcome them by using
alternative representations. No new results or insights are provided.
\tbr
\end{abstract}

\section{Introduction}
\label{sec:introduction}

In computer-aided design (CAD) as well as other domains like artificial
intelligence or combinatorics, many problems can be formulated as Boolean
functions. These Boolean functions then can be represented symbolically using the
\textit{Binary Decision Diagrams} (\textit{BDDs}) introduced by Lee\cite{LEE59}
and Akers\cite{AKERS78} in 1959 and 1978, respectively.

In 1986 Bryant\cite{BRYANT86} published a paper of high impact, describing a
\textit{reduced} and \textit{ordered} class of BDDs, so-called \textit{ROBDDs}
(\textit{Reduced Ordered} BDDs) but in literature mostly referred to as
\textit{OBDDs}. OBDDs represent Boolean functions in a \textit{canonical}
form. This results in a compact form and very efficient tests for some properties,
e.g. satisfiability or equivalence\cite{BRYANT86}.

Already in the inital paper describing OBDDs, the importance of the variable
ordering was highlighted. Only if the variable ordering is chosen properly, it
will result in a small graph (OBDD) and, in turn, in more efficient testing.

Nonetheless representing Boolean functions using OBDDs has some drawbacks. Classes
of Boolean functions exist for which the size of the graph grows exponentially,
e.g. integer multiplication~\cite{BRYANT86, BRYANT91, WOELFEL01}.

These limitations of OBDDs motivated researchers to introduce other
representations which e.g. relax the ordering~\cite{BRYANT95} or change the
interpretation of nodes~\cite{BRYANT95}~\cite{ANDERSEN97}. Such representations
may be beneficial for some classes of Boolean functions but mostly suffer from
losing the canonical form and hence the desirable properties OBDDs are famous for.

This paper serves as survey on symbolic Boolean function manipulation using OBDDs.
The following section describes the representation of Boolean functions through
BDDs as well as OBDDs. It also highlights the advantages and disadvantages of the
OBDD representation. The third section describes operations which can be applied
to OBDDs efficiently. No source code is provided but those can be looked up in
the corresponding papers easily (references are provided for each algorithm).
Section~\ref{sec:implementation-aspects} describes aspects one has to take into
account when implementing an OBDD package, such as what data structures to use
or how to deal with the problem of variable ordering. Limitations of the OBDD
representation are described in the subsequent section. The last section
discusses alternative representations to possibly overcome the limitations of
the OBDD representation.
\tbr

\section{Representation Basics}
\label{sec:representation-basics}

\tbd

\subsection{Reducing and Ordering BDDs}
\label{subsec:reducing-and-ordering-bdds}

\tbd

\section{Efficient Operations}
\label{sec:efficient-operations}

\tbd

\section{Implementational Aspects}
\label{sec:implementation-aspects}

\tbd

\subsection{Dynamic Variable Ordering}
\label{subsec:dynamic-variable-ordering}

\tbd

\section{Limitations}
\label{sec:limitations}

\tbd

\section{Alternative Representations}
\label{sec:alternative-representations}

\tbd

\section{Citations (temporary)}

\noindent
\par
BRACE90\cite{BRACE90}, BRYANT92\cite{BRYANT92},
\par
RUDELL93\cite{RUDELL93}, BOLLIG96\cite{BOLLIG96}


% balance columns on last page
\balance

\bibliographystyle{abbrv}
\bibliography{paper}

\end{document}
